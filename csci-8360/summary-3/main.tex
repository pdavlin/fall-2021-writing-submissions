% BEGIN TEMPLATE
\documentclass[11pt]{article}
\usepackage{graphicx}
\usepackage{hyperref} 
\usepackage{xcolor}
\usepackage{nameref}
\usepackage{listings}
\usepackage{float}
\usepackage[title]{appendix}
\usepackage[ruled]{algorithm2e}
\graphicspath{ {../../images/} }
\bibliographystyle{acm}
% CHANGE THESE
\newcommand{\courseListing}{CSCI 8360}
\newcommand{\courseName}{Machine Learning for Text}
\newcommand{\assignmentTitle}{Summary \#3}
\newcommand{\assignmentSubtitle}{Python K-Means}
\usepackage{geometry}
\geometry{margin=1in}

\hypersetup{
    colorlinks,
    linkcolor={red!50!black},
    citecolor={blue!50!black},
    urlcolor={blue!80!black}
}
\urlstyle{same}
\definecolor{codegreen}{rgb}{0,0.6,0}
\definecolor{codegray}{rgb}{0.5,0.5,0.5}
\definecolor{codepurple}{rgb}{0.58,0,0.82}
\lstdefinestyle{mystyle}{
    commentstyle=\color{codegreen},
    keywordstyle=\color{magenta},
    numberstyle=\tiny\color{codegray},
    stringstyle=\color{codepurple},
    basicstyle=\ttfamily\footnotesize,
    breakatwhitespace=false,         
    breaklines=true,                 
    captionpos=b,                    
    keepspaces=true,                 
    numbers=left,                    
    numbersep=5pt,                  
    showspaces=false,                
    showstringspaces=false,
    showtabs=false,                  
    tabsize=2
}

\lstset{style=mystyle}

\begin{document}
  \input{../../templates/titlepage.tex}
  \graphicspath{{./images/}}
\newpage
For this assignment, the primary task was to implement and observe the K-Means algorithm in Python.
In doing so, students were permitted to use any libraries necessary to implement the algorithm, with the intent being to observe the impact of various parameters on the algorithm.
With this in mind, a logical first step was to import the scikit-learn (\lstinline{sklearn}) package into the script and import its \lstinline{KMeans} subpackage.
This feels a bit like skipping to the end, but in the scope of the assignment was a sensible approach--the K-Means implementation in \lstinline{sklearn} is quite long--keeping in mind that the intent was to observe the algorithm in practice.
Handily, \lstinline{sklearn} also has the 20 newsgroup dataset built in as well.

Like any machine learning implementation, the first and most intensive part of this assignment was developing an understanding of what the data is and how to organize it for use with the applied model.
Approaching this, the first goal was to vectorize the incoming text for use in the K-Means model.
Fortunately, \lstinline{sklearn} contains a \lstinline{TfidfVectorizer} used to convert 

\newpage
\begin{appendices}
\section{K-Means} \label{outputs}
\newpage
\section{Complete Code Listing} \label{codelist}


\end{appendices}
  
\end{document}